\documentclass[a4paper,final,12pt]{article}
\usepackage[utf8]{inputenc}
\usepackage[T1]{fontenc}
\usepackage{times}
\usepackage[english,brazil]{babel}
%\usepackage{mathbbol}
\usepackage{amssymb,amsfonts}
\usepackage{layouts}
\usepackage[top=3cm,bottom=3cm,left=3cm,right=3cm]{geometry}
\usepackage[alf,bibjustif]{abntex2cite}
\usepackage{indentfirst}
\usepackage{url}
\baselineskip=18pt
\setlength{\parskip}{6pt}
\setlength{\parindent}{24pt}

\title{Experiências com provador de teoremas}
\author{José Marcos da Silva Leite}

\begin{document}
	\maketitle
	
	\section{Lógica Proposicional}
	\subsection{Sintaxe}
	\indent Chamemos um conjunto enumerável $P = \{p, q, r, ..., p_{1}, q_{1}, ...\}$ de \textbf{símbolos proposicionais}. \\
	\indent Chamemos o conjunto $O = \{\lnot, \land, \lor, \to, \leftrightarrow \}$ de \textbf{operadores proposicionais}. \\
	\indent Chamemos o conjunto $S = \{(, )\}$ de \textbf{sinais de pontuação}. \\
	\indent Uma \textbf{fórmula proposicional} é qualquer sequência finita de $P \cup O \cup S$. \\
	\indent Definimos o conjunto de \textbf{Fórmulas Bem Formadas} $FBF_{LP}$ recursivamente:
	\begin{enumerate}
		\item se $\varphi \in P$, então $\varphi \in FBF_{LP}$
		\item se $\varphi \in FBF_{LP}$, então $\lnot\varphi \in FBF_{LP}$
		\item se $\varphi \in FPF_{LP}$, $\psi \in FPF_{LP}$ e $* \in O \setminus \{\lnot\}$, então $(\varphi * \psi) \in FBF_{LP},$
	\end{enumerate}

	\subsection{Semântica}
	Definimos uma função $\mathbb{V}_{0} : P \to \{V, F\}$. \\
	\indent Definimos uma função $\mathbb{V} : FBF_{LP} \to \{V, F\}$:
	\begin{enumerate}
		\item $\mathbb{V}(\varphi) = \mathbb{V}_{0}(\varphi)$, se $\varphi \in P$
		\item $\mathbb{V}(\lnot\varphi) = V$, somente se, $\mathbb{V}(\varphi) = F$
		\item $\mathbb{V}(\varphi \land \psi)$, somente se, $\mathbb{V}(\varphi) = V$ e $\mathbb{V}(\psi) = V$
		\item $\mathbb{V}(\varphi \lor \psi)$, somente se, $\mathbb{V}(\varphi) = V$ ou $\mathbb{V}(\psi) = V$
		\item $\mathbb{V}(\varphi \to \psi) = V$, somente se, $\mathbb{V}(\varphi) = F$ ou $\mathbb{V}(\psi) = V$
		\item $\mathbb{V}(\varphi \leftrightarrow \psi) = V$, somente se, $\mathbb{V}(\varphi) = \mathbb{V}(\psi)$
	\end{enumerate}
	
	Se existe uma valoração $\mathbb{V} : FBF_{LP} \to \{V, F\}$ tal que $\mathbb{V}(\varphi) = V$, então $\varphi$ é \textbf{satisfatível}. Se não existe, $\varphi$ é \textbf{insatisfatível}.
	\subsection{Resolução}
	Um \textbf{literal} é um simbolo proposicional ou sua negação. \\
	\indent Uma \textbf{cláusula} é uma disjunção de literais. \\
	\indent Uma fórmula está na \textbf{Forma Normal Conjuntiva}(ou \textbf{FNC}) somente se esta for uma conjunção de cláusulas. \\
	\indent Um \textbf{axioma} é uma fórmula bem formada que, geralmente, é aceita como verdade. \\
	\indent Uma \textbf{regra de inferência} é uma forma de obtenção de formulas a partir de um conjunto de fórmulas.
	
	\section{Automação de prova de teoremas}
	\subsection{POVO QUE INVENTOU}
	\subsection{KSP}
	
	\section{Resultados}
	\section{Conclusão}
	
\end{document}